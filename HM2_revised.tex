\documentclass{article}

\usepackage{fancyhdr}
\usepackage{extramarks}
\usepackage{amsmath}
\usepackage{amsthm}
\usepackage{amsfonts}
\usepackage{tikz}
\usepackage[plain]{algorithm}
\usepackage{algpseudocode}
\usepackage{soul}

\usetikzlibrary{automata,positioning}

%
% Basic Document Settings
%

\topmargin=-0.45in
\evensidemargin=0in
\oddsidemargin=0in
\textwidth=6.5in
\textheight=9.0in
\headsep=0.25in

\linespread{1.1}

\pagestyle{fancy}


\renewcommand\headrulewidth{0.4pt}
\renewcommand\footrulewidth{0.4pt}

\setlength\parindent{0pt}

%
% Create Problem Sections
%

\newcommand{\enterProblemHeader}[1]{
	\nobreak\extramarks{}{Problem \arabic{#1} continued on next page\ldots}\nobreak{}
	\nobreak\extramarks{Problem \arabic{#1} (continued)}{Problem \arabic{#1} continued on next page\ldots}\nobreak{}
}

\newcommand{\exitProblemHeader}[1]{
	\nobreak\extramarks{Problem \arabic{#1} (continued)}{Problem \arabic{#1} continued on next page\ldots}\nobreak{}
	\stepcounter{#1}
	\nobreak\extramarks{Problem \arabic{#1}}{}\nobreak{}
}

\setcounter{secnumdepth}{0}
\newcounter{partCounter}
\newcounter{homeworkProblemCounter}
\setcounter{homeworkProblemCounter}{1}
\nobreak\extramarks{Problem \arabic{homeworkProblemCounter}}{}\nobreak{}

%
% Homework Problem Environment
%
% This environment takes an optional argument. When given, it will adjust the
% problem counter. This is useful for when the problems given for your
% assignment aren't sequential. See the last 3 problems of this template for an
% example.
%
\newenvironment{homeworkProblem}[1][-1]{
	\ifnum#1>0
	\setcounter{homeworkProblemCounter}{#1}
	\fi
	\section{Problem \arabic{homeworkProblemCounter}}
	\setcounter{partCounter}{1}
	\enterProblemHeader{homeworkProblemCounter}
}{
\exitProblemHeader{homeworkProblemCounter}
}

%
% Homework Details
%   - Title
%   - Due date
%   - Class
%   - Section/Time
%   - Instructor
%   - Author
%

\newcommand{\hmwkTitle}{Homework\ \#2}
\newcommand{\hmwkDueDate}{29.03.16}
\newcommand{\hmwkClass}{Non linear models in operations research}
\newcommand{\hmwkClassTime}{}
\newcommand{\hmwkClassInstructor}{}
\newcommand{\hmwkAuthorName}{}

%
% Title Page
%

\title{
	\vspace{2in}
	\textmd{\textbf{\hmwkClass:\ \\
			\hmwkTitle}}\\
	\normalsize\vspace{0.1in}\small{Due\ on\ \hmwkDueDate\ at 23:55pm}\\
	\vspace{0.1in}\large{\textit{\hmwkClassInstructor\ \hmwkClassTime}}
	\vspace{3in}
}

\author{\textbf{Yevgeny Shapiro 306707076, Uri Lerner 066055328}}

\date{}

\renewcommand{\part}[1]{\textbf{\large Part \Alph{partCounter}}\stepcounter{partCounter}\\}

%
% Various Helper Commands
%

% Useful for algorithms
\newcommand{\alg}[1]{\textsc{\bfseries \footnotesize #1}}

% For derivatives
\newcommand{\deriv}[1]{\frac{\mathrm{d}}{\mathrm{d}x} (#1)}

% For partial derivatives
\newcommand{\pderiv}[2]{\frac{\partial}{\partial #1} (#2)}

% Integral dx
\newcommand{\dx}{\mathrm{d}x}

% Alias for the Solution section header
\newcommand{\solution}{\textbf{\large Solution}}

% Probability commands: Expectation, Variance, Covariance, Bias
\newcommand{\E}{\mathrm{E}}
\newcommand{\Var}{\mathrm{Var}}
\newcommand{\Cov}{\mathrm{Cov}}
\newcommand{\Bias}{\mathrm{Bias}}

\begin{document}
	
	\maketitle
	
	\pagebreak
	
	\begin{homeworkProblem}
		
	\part
	\\
	\\
	Let us find the general formula for eigenvalues and eigenvectors for the matrix of the form: \\

	
	\[
	M=\begin{bmatrix}
	a+1       & \frac{1}{2}b \\
	\frac{1}{2}b      & 1
	\end{bmatrix}
	\]
	First let us find the eigenvalues solving: 
	\[
	|\lambda{}I - M| = 
		\begin{vmatrix}
		\lambda - (a+1)       & -\frac{1}{2}b \\
		-\frac{1}{2}b      & \lambda - 1
		\end{vmatrix}
		= (\lambda - a - 1)(\lambda - 1)-\frac{1}{4}b = 0
	\]
	Solving the equation we get: 
	\[
	\lambda^{2}-(2+a)\lambda - (\frac{1}{b^2} + a - 1) = 0
	\]
	
	Finding the roots: \\
	\[
	\Delta = b^2 - 4ac = (2+a)^2 - 4(a + 1 - \frac{1}{4}b^2) = (a^2 + b^2)\\
	\]
	\[
		\lambda_{1} = \frac{1}{2}(\sqrt{a^2 + b^2} + a + 2)
	\]

	\[
		\lambda_{2} = \frac{1}{2}(-\sqrt{a^2 + b^2} + a + 2)
	\]
		
	Now let us solve the equation for $\lambda_1$ \\
	\[
	|\lambda_{1} I - M| \cdot{}x = 0
	\]
		\[
		|\lambda{}I - M| = 
		\begin{bmatrix}
		\lambda - (a+1)       & \frac{1}{2}b \\
		\frac{1}{2}b      & \lambda - 1
		\end{bmatrix}
=
		\begin{bmatrix}
		\frac{1}{2}(\sqrt{a^2 + b^2} + a + 2) - a- 1)      & \frac{1}{2}b \\
		\frac{1}{2}b      & \frac{1}{2}(\sqrt{a^2 + b^2} + a + 2)  - 1
		\end{bmatrix}
		\]
			
	This leads to the following equations:\\
	\begin{equation}
	(\sqrt{a^2 + b^2} - a)\cdot{}x_1 - b\cdot{}x_2 = 0
	\end{equation}
	\begin{equation}
	b\cdot{}x_1 - (\sqrt{a^2 + b^2} + a)\cdot{}x_2 = 0
	\end{equation}
	Simplifying equation 2: 
	\[
	x_1 = \frac{\sqrt{a^2 + b^2} + a}{b}\cdot{}x_2
	\]
	Which corresponds to the eigenvector: 
	\[
	\begin{bmatrix}
	\frac{\sqrt{a^2 + b^2} + a}{b} &
	 1
	\end{bmatrix}^T
	\]
	Assigning these values to equation (1) results in 0 = 0, hence, this is a valid solution. \\
	\\
	Similarly, for $\lambda_2$ we get:
			\[
			\begin{bmatrix}
			\frac{-\sqrt{a^2 + b^2} + a}{b} &
			1
			\end{bmatrix}^T
			\]
	Thus, the eigenvalues and eigenvectors can be represented as: 
	\[
	D = 	\begin{bmatrix}
	\frac{1}{2}(\sqrt{a^2 + b^2} + a + 2)       & 0 \\
	0      & \frac{1}{2}(-\sqrt{a^2 + b^2} + a + 2)
	\end{bmatrix}
	\]
	\[
	U = 	\begin{bmatrix}
			\frac{\sqrt{a^2 + b^2} + a}{b}       & \frac{-\sqrt{a^2 + b^2} + a}{b} \\
			1      & 1
		\end{bmatrix}
	\]
	
	\part
	\\
	Given 
		\[
	P = \begin{bmatrix}
		-4      & 2 \\
		2      & -2
		\end{bmatrix}
		\]
	We would like to use the general formula from part A. Thus, we need to adjust the matrix $P$ to the form of matrix $M$.\\
			\[
			P = -2\cdot{}\begin{bmatrix}
			2      & -1 \\
			-1      & 1
			\end{bmatrix}
			\]
			 
	Now we have a similar form, with $ a = 1, b = -2$\\
	
		\[
		D_{P} = 	-2 \cdot{}\begin{bmatrix}
		\frac{1}{2}(\sqrt{a^2 + b^2} + a + 2)       & 0 \\
		0      & \frac{1}{2}(-\sqrt{a^2 + b^2} + a + 2)
		\end{bmatrix}
		=
		-2 \cdot{}
		\begin{bmatrix}
		\frac{1}{2}(\sqrt{5} + 3)       & 0 \\
		0      & \frac{1}{2}(-\sqrt{5} + 3)  
		\end{bmatrix}  
		= 
		\begin{bmatrix}
		-(\sqrt{5} + 3)       & 0 \\
		0      & (\sqrt{5} - 3)  
		\end{bmatrix}
		\]
However, the eigenvectors are defined up to a constant, and do not change under scalar multiplication.		
		
		
		
		\[
				U_P = 	\begin{bmatrix}
				\frac{\sqrt{a^2 + b^2} + a}{b}       & \frac{-\sqrt{a^2 + b^2} + a}{b} \\
				1      & 1
		\end{bmatrix}
		 = \begin{bmatrix}
		\frac{\sqrt{5} + 1}{-2}  & \frac{-\sqrt{5} + 1}{-2} \\
		1      & 1
		\end{bmatrix}
		\]

		
\part
\\
Given 
		\[
	A =  \begin{bmatrix}
		1      & -2 & 3 & -1 \\
		-2    & 1 & -1 & -2
		\end{bmatrix}
		\]
In order to obtain the singular values of $A$ we can use the square root of the eigenvalues of $AA^T$:\\
		\[
		AA^T =  \begin{bmatrix}
		1      & -2 & 3 & -1 \\
		-2    & 1 & -1 & -2
		\end{bmatrix}
		\cdot{} \begin{bmatrix}
			1      & -2 \\
			-2    & 1 \\
			3 & -1 \\
			-1 & -2 
		\end{bmatrix}
		 =
		  \begin{bmatrix}
		  15    & -5 \\
		  -5    & 10 \\
		  \end{bmatrix}
		\]
In order to use the general formula, we can divide the matrix by 10 and get:\\
\[
 AA^T =	10\cdot{} \begin{bmatrix}
		1.5   & -0.5 \\
		-0.5    & 1 \\
	\end{bmatrix}
\]
With $a = 0.5, b = -1$


	\[
	D_{AA^T} = 	10 \cdot{}\begin{bmatrix}
	\frac{1}{2}(\sqrt{a^2 + b^2} + a + 2)       & 0 \\
	0      & \frac{1}{2}(-\sqrt{a^2 + b^2} + a + 2)
	\end{bmatrix}
	=
	5 \cdot{}\begin{bmatrix}
	(\frac{\sqrt{5}}{2} + 2.5)       & 0 \\
	0      & (-\frac{\sqrt{5}}{2} + 2.5)
	\end{bmatrix}
	 = 
	 \begin{bmatrix}
	 \frac{5(\sqrt{5} + 5)}{2}       & 0 \\
	 0      &  \frac{5(-\sqrt{5} + 5)}{2} 
	 \end{bmatrix}
	\]

To find the singular values of A, we need to take the square root of the eigenvalues of $AA^T$:

	\[
	\lambda_1 = \sqrt{\frac{5(\sqrt{5} + 5)}{2}}
	\]
	\[
	\lambda_2 = \sqrt{-\frac{5(\sqrt{5} + 5)}{2}}
	\]
\end{homeworkProblem}

\begin{homeworkProblem}
Going left to right, 
Proving inequality 1: $\frac{1}{\sqrt{n}}\lVert x \rVert_2  \leq  \lVert x \rVert_{\infty}$\\
\[
\mathbf{\frac{1}{\sqrt{n}}\lVert x \rVert_2} = \frac{1}{\sqrt{n}}\sqrt{\sum_{i = 1}^{n}x^2} \leq  \frac{1}{\sqrt{n}}\cdot{}\sqrt{n\cdot{}\max_{i}x_i^2} = \frac{1}{\sqrt{n}}\cdot{}\sqrt{n}\cdot{}\sqrt{\max_{i}x_i^2} =  \max_i{}|x_i| = \mathbf{\lVert x \rVert_{\infty}
}\]
hence 
\[
\frac{1}{\sqrt{n}}\lVert x \rVert_2  \leq  \mathbf{\lVert x \rVert_{\infty}
}\]
\\
Proving inequality 2: $\lVert x \rVert_{\infty}  \leq  \lVert x \rVert_{2}$\\
\\
Let us define $\alpha$ as the sum of all elements except $\arg\max_i x_i^2$.
thus:
\[
\lVert x \rVert_2  = 
\sqrt{\sum_{i=1}^n x_i^2} = \sqrt{\max_{i}x_i^2 + \alpha}
\]
Because $\alpha \geq 0$ we have:
\[
\sqrt{\max_{i}x_i^2 + \alpha} \geq \sqrt{\max_{i}x_i^2} =  \lVert x \rVert_{\infty}
\]
Hence: 
\[
\lVert x \rVert_{\infty}  \leq  \lVert x \rVert_{2}
	\]
Proving inequality 3: $\lVert x \rVert_{2}  \leq  \lVert x \rVert_{1}$\\
According to the triangle inequality, the sum of norms is less or equal then the norm of sums: \\
\[
\lVert x + y \rVert_{2} \leq \lVert x \rVert_{2} + \lVert y \rVert_{2}
\]
Using a more general form we get:
\[
\lVert x \rVert_{2} = \sqrt{\sum_{i=1}^n (x_i)^2} \leq \sum_{i=1}^n \sqrt{(x_i)^2} = \lVert x \rVert_{1}
\]
\\
Proving inequality 4: $\lVert x \rVert_{1}  \leq \sqrt{n} \lVert x \rVert_{2}$\\
Based on the of Cauchy-Schwarz inequality: 
$\lvert x\cdot{}y \rvert \leq  \lVert x \rVert_2 \lVert y \rVert_2$\\
We can define $y = sign(x)$ of length n, such that $\lVert y \rVert_2 \leq \sqrt{n}$ (since for zeros $sign(x)=0$). \\Thus: \\
\[
\lVert x \rVert_1 = |x \cdot{}sign(x)|  \leq  \lVert x \rVert_2 \lVert y \rVert_2  \leq \sqrt{n} \lVert x \rVert_2 
\]
Hence: 
\[
\lVert x \rVert_1 \leq  \sqrt{n} \lVert x \rVert_2 
\]

Proving inequality 5: $\sqrt{n} \lVert x \rVert_2  \leq n \lVert x \rVert_{\infty}$\\
\\
In inequality 1 we proved that: $\frac{1}{\sqrt{n}}\lVert x \rVert_2  \leq  \lVert x \rVert_{\infty}$\\
multiplying by $n$ we get $\sqrt{n} \lVert x \rVert_2  \leq n \lVert x \rVert_{\infty}$
\end{homeworkProblem}

\begin{homeworkProblem}
Using the Cauchy-Schwarz formula with $y = sign(x)$ we have:\\
 \[
 |<x,y>| \leq \lVert x \rVert_2 \cdot{} \lVert y \rVert_2
 \]
 \[
 \sum_{i=1}^n x\cdot sign(x) \leq \lVert x \rVert_2 \cdot{} \sum_{i=1}^n (sign(x))^2
 \]
 On the right side we use the fact that the sum of $sign(x)$ counts all the non zero elements in x, and equals to $\lVert x \rVert_0$\\
 On the eleft side we use the fact that $x\cdot sign(x) = |x|$\\
 Thus: 
  \[
  \lVert x \rVert_1 \leq \lVert x \rVert_2 \cdot{} \lVert x \rVert_0
  \]
  \\Using this equation it can we can show the following:
\[
\frac{\lVert x \rVert_1^2}{\lVert x \rVert^2_2}  \leq
\frac{\lVert x \rVert_2^2 \lVert x \rVert_0}{\lVert x \rVert_2^2}^2 = \lVert x \rVert_0
\]
A vector that satisfies equality: $\alpha \cdot e_i$. In this case $\frac{\lVert x \rVert_1^2}{\lVert x \rVert_2^2} = \frac{\alpha^2}{\alpha^2} = 1 = \lVert x \rVert_0 $. 
\end{homeworkProblem}

\begin{homeworkProblem}
	\part
	\\
	(1)
	\\
		Proving: if A is an orthogonal matrix then $\phi{}(A) = 0$ \\
	If A is an orthogonal matrix, $A^TA = I$, hence for each row individually we have: 
		\[
		A_{i}^T{}A_j = 
		\begin{cases}
		1 & \text{if } i = j \\
		0 & \text{if } i \neq j
		\end{cases} 
		\]
	Using the definition of the function $\phi$ and the definition of $\delta$, we see that for $|A_{i}^T{}A_j - \delta_{i,j}|$ \\
	\[
	|A_{i}^T{}A_j - \delta_{i,j}| = 
	\begin{cases}
	1& \text{if } i = j \\
	0& \text{if } i \neq j
	\end{cases}
	 - 
	\begin{cases}
	1& \text{if } i = j \\
	0& \text{if } i \neq j
	\end{cases}
	 =
	\begin{cases}
	0& \text{if } i = j \\
	0& \text{if } i \neq j
	\end{cases}
	\]
	
	\textbf{Thus the sum over $i,j$ is also zero.} \\
	\\
	(2) 
	Proving: if $\phi{}(A) = 0$ then A is an orthogonal matrix  \\
	Proving by contradiction, let us assume that $\phi{}(A) = 0$, but A is not an orthogonal matrix.\\ 
	A non orthogonal matrix will be characterized by at least one incident of the following two cases\\
	\\
	Case 1: \\
	Matrix A has at least one pair of vectors which are not orthogonal i.e. $\exists{}\hat{i},\hat{j}$ such that $\hat{i} \neq\hat{j}$ but $A_{\hat{i}}^T A_{\hat{j}} \neq 0$\\ 
	In this case, at least one of the elements included in the summation within the function $\phi$ will not be equal 0 (as $\delta_{ij} = 0$ for this element). Individual elements in the summation 
	cannot cancel each other, as each element is taken as an absolute value.\\
	Eventually, $\phi{}(A)$ will not be equal zero.\\
	\\
	Case 2: \\
	Matrix A has at least one vector which is not normalized i.e. $\exists{}\hat{i}$ such that $A_{\hat{i}}^T A_{\hat{i}} \neq 1$\\
		In this case, at least one of the elements included in the summation within the function $\phi$ will not be equal 0 (as $\delta_{ij} = 1$ for this element, but $A_{\hat{i}}^T A_{\hat{i}} - \delta_{ij} \neq 0$). Individual elements in the summation 
		cannot cancel each other, as each element is taken as an absolute value.\\
		Eventually, $\phi{}(A)$ will not be equal zero.\\
	\\
	As we show, there cannot be any non orthogonal matrix $A$ that will result in $\phi{}(A) = 0$, we \textbf{contradict our initial assumption}.	\\
	\\
\textbf{Thus}, we proved both directions, hence $\phi{}(A) = 0$ if and only if A is an orthogonal matrix. \\
\\
	\part \\
	\\
	Running the commands:\\
	A = [−3 1 1 ; 2 1 −1; 2 3 2 ];\\
	B = orth(A)+0.1*ones(size(A));\\
	orthogonality\_measure(B)\\
	The output is: \\
	ans =\\
	1.0460\\
	\\ 
	For orth(A) we got a value infinitesimally close to zero, however, $B = orth(A) + \epsilon$.
\end{homeworkProblem}
	\pagebreak
	


\end{document}